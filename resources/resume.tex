%%%%%%%%%%%%%%%%%%%%%%%%%%%% Document Setup %%%%%%%%%%%%%%%%%%%%%%%%%%%%

% Don't like 10pt? Try 11pt or 12pt
\documentclass[10pt]{article}

% This is a helpful package that puts math inside length specifications
\usepackage{calc}

% Layout: Puts the section titles on left side of page
\reversemarginpar

% Changes the layout for CV style section headings as marginal notes.
% The margin widths and section title widths can be easily adjusted.
\usepackage[paper=letterpaper,
            marginparwidth=1in,     % Length of section titles
            marginparsep=.05in,       % Space between titles and text
            margin=0.5in,               % 1 inch margins
            includemp]{geometry}

%% Get rid of indenting throughout entire document
\setlength{\parindent}{0in}

%% This gives us fun enumeration environments. compactitem will be nice.
\usepackage{paralist}

\pagestyle{empty}      % No page numbers

% Finally, give us PDF bookmarks
\usepackage{color,hyperref}
\definecolor{darkblue}{rgb}{0.0,0.0,0.3}
\hypersetup{colorlinks,breaklinks,
            linkcolor=darkblue,urlcolor=darkblue,
            anchorcolor=darkblue,citecolor=darkblue}

%%%%%%%%%%%%%%%%%%%%%%%% End Document Setup %%%%%%%%%%%%%%%%%%%%%%%%%%%%

%%%%%%%%%%%%%%%%%%%%%%%%%%% Helper Commands %%%%%%%%%%%%%%%%%%%%%%%%%%%%

% The title (name) with a horizontal rule under it
%
% Usage: \makeheading{name}
%
% Place at top of document. It should be the first thing.
\newcommand{\makeheading}[1]%
        {\hspace*{-\marginparsep minus \marginparwidth}%
         \begin{minipage}[t]{\textwidth+\marginparwidth+\marginparsep}%
                {\Large \bfseries #1}\\[-0.8\baselineskip]%
                 \rule{\columnwidth}{0.25pt}%
         \end{minipage}}

% The section headings
%
% Usage: \section{section name}
%
% Follow this section IMMEDIATELY with the first line of the section
% text. Do not put whitespace in between. That is, do this:
%
%       \section{My Information}
%       Here is my information.
%
% and NOT this:
%
%       \section{My Information}
%
%       Here is my information.
%
% Otherwise the top of the section header will not line up with the top
% of the section. Of course, using a single comment character (%) on
% empty lines allows for the function of the first example with the
% readability of the second example.
\renewcommand{\section}[2]%
        {\pagebreak[3]\vspace{1.3\baselineskip}%
         \phantomsection\addcontentsline{toc}{section}{#1}%
         \hspace{0in}%
         \marginpar{
         \raggedright \small \scshape #1}#2}

% An itemize-style list with lots of space between items
\newenvironment{outerlist}[1][\enskip\textbullet]%
        {\begin{itemize}[#1]}{\end{itemize}%
         \vspace{-.6\baselineskip}}

% An itemize-style list with little space between items
\newenvironment{innerlist}[1][\enskip\textbullet]%
        {\begin{compactitem}[#1]}{\end{compactitem}}

% To add some paragraph space between lines.
% This also tells LaTeX to preferably break a page on one of these gaps
% if there is a needed pagebreak nearby.
\newcommand{\blankline}{\quad\pagebreak[2]\vspace{-0.3\baselineskip}}

% For \email{ADDRESS}, links ADDRESS to the url mailto:ADDRESS
\providecommand*\email[1]{\href{mailto:#1}{#1}}

%%%%%%%%%%%%%%%%%%%%%%%% End Helper Commands %%%%%%%%%%%%%%%%%%%%%%%%%%%

%%%%%%%%%%%%%%%%%%%%%%%%% Begin CV Document %%%%%%%%%%%%%%%%%%%%%%%%%%%%

\begin{document}
\makeheading{Aaron Gable}

\section{Contact Information}
% NOTE: Mind where the & separators and \\ breaks are in the following
%       table.
\begin{tabular}[t]{@{}p{3.45in}p{2.5in}}
\email{aaron@aarongable.com}                  & \href{https://github.com/aarongable}{github.com/aarongable} \\
\href{https://aarongable.com}{aarongable.com} & \href{https://linkedin.com/in/agable}{linkedin.com/in/agable} \\
\end{tabular}

\section{Objective}
    A senior software engineering technical lead or individual contributor role
    at a company innovating in their field, with a special interest in open source.

\section{Work Experience}
    \textbf{Google, Inc.}
    \hfill \textbf{2012 -- Present} \\
    \textit{Chrome Browser Infrastructure and Operations}

    \blankline % This separates entries in the list

    \textbf{Senior Software Engineer}
    \hfill \textbf{2019 -- Present} \\
    \textit{Chrome Browser Core EngProd, Tech Lead and Manager}
    \begin{innerlist}
        \item Founder, manager, and technical lead of a new team of five software engineers.
        \item Owned configuration, performance, and capacity management of the largest open-source
            pre- and post-submit continuous integration build and test automation system in the world:
            \begin{innerlist}
                \item Pre-submit: 1,500 commit attempts per day, 900 hours of tests per attempt
                \item Post-submit: 530+ distinct platform configurations, 
                \item 12,000 hosts including 5 locations with physical hardware and the cloud
            \end{innerlist}
        \item Designed and deployed solution to extend the system to cover release branches, raising throughput 11\%.
        \item Reduced pre-submit wall-clock time by 20\% at the median, from 48 to 38 minutes.
    \end{innerlist}

    \blankline % This separates entries in the list

    \textbf{Senior Software Engineer}
    \hfill \textbf{2016 -- 2019} \\
    \textit{Chrome Developer Experience, Individual Contributor and Tech Lead}
    \begin{innerlist}
        \item Technical Lead of the team owning Chrome's developer command-line SDK, code review
            application, code search service, and issue tracking tool.
        \item Led the migration of 700+ repositories to use Gerrit for code reviews, including
            work to bring Gerrit to feature and performance parity with the prior system.
    \end{innerlist}
    
    \blankline % This separates entries in the list

    \textbf{Software Engineer III}
    \hfill \textbf{2014 -- 2016} \\
    \textit{Chrome Infrastructure, Individual Contributor}
    \begin{innerlist}
        \item Developed and launched Monorail, a complete bug and issue tracking application to
            replace Chromium's prior home on Google Codesite.
        \item Built and operated pipeline to migrate 16 projects, 600k issues, 5 million comments,
            and 300k users from Codesite to Monorail in under 4 hours with zero data loss.
        \item Established timeseries monitoring libraries, pipelines, and best practices, and
            drove adoption of monitoring for 5000 physical hosts and 13 AppEngine apps.
    \end{innerlist}
    
    \blankline % This separates entries in the list

    \textbf{Software Engineer II}
    \hfill \textbf{2012 -- 2014} \\
    \textit{Chrome Infrastructure, Individual Contributor}
    \begin{innerlist}
        \item Designed and developed Recipes, a remote execution system to make Chrome's continuous
            integration processes self-contained, dynamic, and testable.
        \item Worked with developers and release managers to design and ship tools which enabled
            them to interact with Git-based replicas of Chromium's Subversion repository.
        \item Migrated the Chromium repository and all 200+ of its dependencies from Subversion
            to Git. Final switch-over accomplished with zero hours of developer down-time.
    \end{innerlist}

\section{Other Experience}
    \textbf{Computer Science Instructor, Xavier University of Louisiana}
    \hfill \textbf{Fall 2018} \\
    \textit{Googler In Residence Program}
    \begin{innerlist}
        \item Developed curriculum, lectures, homework, quizzes, and exams for \emph{Introduction to Computer Science}.
        \item Taught 47 students CS and programming in Python, up through nested data structures and recursion.
        \item Conducted interview practice sessions, leading to 11 students getting internships at top tech companies.
    \end{innerlist}

\section{Skills}
    Fluency: Python, Git, Reliability best practices, Distributed build and test \\
    Experience: Go, C++, JavaScript, Bash, SQL, Prolog, \LaTeX

\section{Education}
    \textbf{Harvey Mudd College}, Claremont, CA
    \hfill \textbf{2008 -- 2012} \\
    \textit{Bachelor of Science, Computer Science and Mathematics}

\end{document}

%%%%%%%%%%%%%%%%%%%%%%%%%% End CV Document %%%%%%%%%%%%%%%%%%%%%%%%%%%%%

%----------------------------------------------------------------------%
% The following is copyright and licensing information for
% redistribution of this LaTeX source code; it also includes a liability
% statement. If this source code is not being redistributed to others,
% it may be omitted. It has no effect on the function of the above code.
%----------------------------------------------------------------------%
% Copyright (c) 2007, 2008, 2009, 2010, 2011 by Theodore P. Pavlic
% Some modifications made by Aaron Gable, 2011, 2019.
%
% Unless otherwise expressly stated, this work is licensed under the
% Creative Commons Attribution-Noncommercial 3.0 United States License. To
% view a copy of this license, visit
% http://creativecommons.org/licenses/by-nc/3.0/us/ or send a letter to
% Creative Commons, 171 Second Street, Suite 300, San Francisco,
% California, 94105, USA.
%
% THE SOFTWARE IS PROVIDED "AS IS", WITHOUT WARRANTY OF ANY KIND, EXPRESS
% OR IMPLIED, INCLUDING BUT NOT LIMITED TO THE WARRANTIES OF
% MERCHANTABILITY, FITNESS FOR A PARTICULAR PURPOSE AND NONINFRINGEMENT.
% IN NO EVENT SHALL THE AUTHORS OR COPYRIGHT HOLDERS BE LIABLE FOR ANY
% CLAIM, DAMAGES OR OTHER LIABILITY, WHETHER IN AN ACTION OF CONTRACT,
% TORT OR OTHERWISE, ARISING FROM, OUT OF OR IN CONNECTION WITH THE
% SOFTWARE OR THE USE OR OTHER DEALINGS IN THE SOFTWARE.
%----------------------------------------------------------------------%